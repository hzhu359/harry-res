%----------------------------------------------------------------------------------------
%	SECTION TITLE
%----------------------------------------------------------------------------------------

\cvsection{Projects}

%----------------------------------------------------------------------------------------
%	SECTION CONTENT
%----------------------------------------------------------------------------------------

\begin{cventries}

%------------------------------------------------

% \cventry
% {Team Member} % Role
% {Yelp Dataset Machine Learning Project} % Event
% {Georgia Tech, Atlanta, GA} % Location
% {May 2017 - Jul. 2017} % Date(s)
% { % Description(s)
% \begin{cvitems}
% \item {Used natural language processing to train machine learners to predict star rating of Yelp reviews based on text sentiment.}
% \item {To read the final report: \url{https://rebrand.ly/yelpreport}}
% \end{cvitems}
% }

%------------------------------------------------

% \cventry
% {Team Member - Augmented Reality Experiences} % Role
% {Vertically Integrated Projects} % Event
% {Georgia Tech, Atlanta, GA} % Location
% {Jan. 2016 - May 2017} % Date(s)
% { % Description(s)
% \begin{cvitems}
% \item {Worked with Argon, an augmented reality browser available on iOS.}
% \item {Created tutorials and sample code to introduce new users to ArgonJS, a Javascript framework that allows developers to integrate augmented reality features into their websites.}
% \end{cvitems}
% }

%------------------------------------------------

% \cventry
% {Team Member} % Role
% {PearlHacks} % Event
% {Chapel Hill, NC} % Location
% {Feb. 2017} % Date(s)
% { % Description(s)
% \begin{cvitems}
% \item {Used Wit.ai to begin building a chat bot that gives music recommendations based on Spotify's API.}
% \end{cvitems}
% }

% %------------------------------------------------

\cvproject
{Team Member} % Role
{
    \href{https://github.com/RodrigoDLPontes/visualization-tool}
    {Data Structures and Algorithms Visualization Tool}
} % Event
{} % Location
{September 2020} % Date(s)
{ % Description(s)
\begin{cvitems}
\item Fixed algorithm definitions in the open-source CS 1332 visualization tool
used by 600+ students to visualize data structures and algorithms.
\item Implemented fundamental visualization using JavaScript, React,
and the University of San Francisco animation API.
\item Improved a codebase that is known to be especially effective
in enabling students to study concepts and assisting teaching assistants and professors to teach concepts.
\end{cvitems}
}

\cvproject
{Team Member} % Role
{
    \href{https://github.com/BadGuy-1863/HackGT7}
    {Squiggle}
} % Event
{} % Location
{October 2020} % Date(s)
{ % Description(s)
\begin{cvitems}
    \item Implemented a front-end interface for a restaurant load-balancing web application
    during HackGT7.
    \item Employed React, \texttt{create-react-app}, Typescript, TSX, and SCSS to 
    create and style reusable React components for use in constructing the interface.
\end{cvitems}
}

\cvproject
{Developer} % Role
{
    \href{https://github.com/hzhu359/fastai-v3}
    {String Instrument Image Classification}
} % Event
{} % Location
{July 2020} % Date(s)
{ % Description(s)
\begin{cvitems}
\item Designed an image recognition machine learning model
that distinguishes between instrument images (of violins and cellos)
using the fastai library on top of PyTorch.
\item Adapted, tuned, and exported the ResNet-34 convolutional neural network for use in transfer learning between the ImageNet dataset and the custom instrument dataset.
\item Deployed the model as a web app using Render and by adapting existing HTML, CSS, and JavaScript code.
\end{cvitems}
}

% \cvproject
% {Developer} % Role
% {Cover Type Prediction} % Event
% {} % Location
% {October 2019} % Date(s)
% { % Description(s)
% \begin{cvitems}
% \item Analyzed and predicted the type of tree in a forest given a list of over 50 attributes
% using the scikit-learn, pandas, and numpy libraries.
% \item Visualized and discovered patterns and clusters within the dataset using the Seaborn library.
% \item Adapted a Random Forest model in order to classify cover types in forests.
% \end{cvitems}
% }

% \cvproject
% {Developer} % Role
% {Georgia Tech Ping-Pong Website} % Event
% {} % Location
% {October 2019} % Date(s)
% { % Description(s)
% \begin{cvitems}
% \item Contributed to a website that tracked when ping-pong tables were in use around the Georgia Tech campus.
% \item Formatted the website using the W3.CSS framework.
% \item Created JavaScript functions to change color of table elements using onclick events.
% \item Completed a front-end implementation of a website that tracked an array of Ping-Pong table locations. 
% \end{cvitems}
% }

% \cvproject
% {Developer} % Role
% {Password Strength Checker} % Event
% {} % Location
% {Dec. 2019} % Date(s)
% { % Description(s)
% \begin{cvitems}
% \item {Used python's re.py regular expression package to verify the length of a password and the presence of certain required characters.}
% \item {Created a customizable function that took in a String password and returned a Boolean value representing whether or not the password passed the set strength rules.}
% \end{cvitems}
% }
% \cvproject
% {Developer} % Role
% {Force of Gravity Calculator} % Event
% {} % Location
% {Sep. 2019} % Date(s)
% { % Description(s)
% \begin{cvitems}
% \item {Engineered a program that takes in two objects' masses and position vectors and returns information about the vector force of gravity acting on a specified object.}
% \item {Translated physical formulas into compilable Java code and used arrays as representations of vectors.}
% \end{cvitems}
% }

%------------------------------------------------

% \cvproject
% {Team Member} % Role
% {Meal Tracker Program} % Event
% {} % Location
% {Apr. 2019} % Date(s)
% { % Description(s)
% \begin{cvitems}
% \item {Designed a program used to track meals, exercise, and caloric intake with Java.}
% \item {Specifically created the toString method and loops to print data from across classes.}
% \end{cvitems}
% }

%------------------------------------------------

% \cvproject
% {Developer} % Role
% {Classical Music Recommendation Program} % Event
% {} % Location
% {Nov. 2018} % Date(s)
% { % Description(s)
% \begin{cvitems}
% \item {Applied conditional logic in Java to recommend relevant classical music based on the specified period, genre, and mood.}
% \end{cvitems}
% }


\end{cventries}